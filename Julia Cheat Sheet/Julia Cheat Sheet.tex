% Mo Jabeen Template for docs 

\documentclass[11pt]{scrartcl} % Font size

\input{structure.tex} % Include the file specifying the document structure and custom commands

%----------------------------------------------------------------------------------------
%	TITLE SECTION
%----------------------------------------------------------------------------------------

\title{	
	\normalfont\normalsize
	\vspace{20pt} % Whitespace
	{\huge Julia Cheat Sheet}\\ % The assignment title
	\vspace{12pt} % Whitespace
	\rule{\linewidth}{2pt}\\ % Thick bottom horizontal rule
}

\author{\small Mo D Jabeen} % Your name

\date{\normalsize\today} % Today's date (\today) or a custom date

\begin{document}

\maketitle % Print the title

\section{General}

\begin{lstlisting}[
	caption=Function definition., % Caption above the listing
	language=python, % Use Julia functions/syntax highlighting
	frame=single, % Frame around the code listing
	showstringspaces=false, % Don't put marks in string spaces
	numbers=left, % Line numbers on left
	numberstyle=\large, % Line numbers styling
	]
	function name()
	code
	end 

	function name()
	r=3

	r,r+2 #Omit the return keyword for tuple return
	end 
\end{lstlisting}

\begin{itemize}
	\item printf for formatted prints uses the module Printf and is macro with synatax @printf	
	\item \%3f : used to show 3 sig fig
	\item \"e : scientific notation
\end{itemize}

\begin{itemize}
	\item index starts at 1 :O
	\item Strings can be indexed like arrays
	\item Combine strings using *
	\item try, catch : for error handling
\end{itemize}

\begin{lstlisting}[
	caption=Dict definition., % Caption above the listing
	language=python, % Use Julia functions/syntax highlighting
	frame=single, % Frame around the code listing
	showstringspaces=false, % Don't put marks in string spaces
	numbers=left, % Line numbers on left
	numberstyle=\large, % Line numbers styling
	]
	d = Dict(1=>"one", 2 => "two")
	d[3] = "three" # Add to the dict

	#Loops and funcs can also be placed in dicts

\end{lstlisting}

\begin{lstlisting}[
	caption=Loop/arrays definition., % Caption above the listing
	language=python, % Use Julia functions/syntax highlighting
	frame=single, % Frame around the code listing
	showstringspaces=false, % Don't put marks in string spaces
	numbers=left, % Line numbers on left
	numberstyle=\large, % Line numbers styling
	]	
	for i in 1:5
	println(i)
	end

	a = collect(1:20) # convert into an array

	a = map((x) -> x^2, [1,3,5,3]) # map performs func on each array element

\end{lstlisting}



\subsection{Subsection}

paragraph

\section{Understanding Text}

\subsubsection{subsubsection}

%----------------------------------------------------------------------------------------
%	FIGURE EXAMPLE
%----------------------------------------------------------------------------------------

% \begin{figure}[h] % [h] forces the figure to be output where it is defined in the code (it suppresses floating)
% 	\centering
% 	\includegraphics[width=0.5\columnwidth]{IMAGE_NAME.jpg} % Example image
% 	\caption{European swallow.}
% \end{figure}

%----------------------------------------------------------------------------------------
% MATH EXAMPLES
%----------------------------------------------------------------------------------------

% \begin{align} 
% 	\label{eq:bayes}
% 	\begin{split}
% 		P(A|B) = \frac{P(B|A)P(A)}{P(B)}
% 	\end{split}					
% \end{align}

%----------------------------------------------------------------------------------------
%	LIST EXAMPLES
%----------------------------------------------------------------------------------------

% \begin{itemize}
% 	\item First item in a list 
% 		\begin{itemize}
% 		\item First item in a list 
% 			\begin{itemize}
% 			\item First item in a list 
% 			\item Second item in a list 
% 			\end{itemize}
% 		\item Second item in a list 
% 		\end{itemize}
% 	\item Second item in a list 
% \end{itemize}

%------------------------------------------------

% \subsection{Numbered List}

% \begin{enumerate}
% 	\item First item in a list 
% 	\item Second item in a list 
% 	\item Third item in a list
% \end{enumerate}

%----------------------------------------------------------------------------------------
%	TABLE EXAMPLE
%----------------------------------------------------------------------------------------

% \section{Interpreting a Table}

% \begin{table}[h] % [h] forces the table to be output where it is defined in the code (it suppresses floating)
% 	\centering % Centre the table
% 	\begin{tabular}{l l l}
% 		\toprule
% 		\textit{Per 50g} & \textbf{Pork} & \textbf{Soy} \\
% 		\midrule
% 		Energy & 760kJ & 538kJ\\
% 		Protein & 7.0g & 9.3g\\
% 		\bottomrule
% 	\end{tabular}
% 	\caption{Sausage nutrition.}
% \end{table}

%----------------------------------------------------------------------------------------
%	CODE LISTING EXAMPLE
%----------------------------------------------------------------------------------------

% \lstinputlisting[
% 	caption=Luftballons Perl Script., % Caption above the listing
% 	label=lst:luftballons, % Label for referencing this listing
% 	language=Perl, % Use Perl functions/syntax highlighting
% 	frame=single, % Frame around the code listing
% 	showstringspaces=false, % Don't put marks in string spaces
% 	numbers=left, % Line numbers on left
% 	numberstyle=\tiny, % Line numbers styling
% 	]{luftballons.pl}

%------------------------------------------------

\end{document}