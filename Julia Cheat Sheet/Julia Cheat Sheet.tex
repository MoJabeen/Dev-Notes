% Mo Jabeen Template for docs 

\documentclass[11pt]{scrartcl} % Font size

%%%%%%%%%%%%%%%%%%%%%%%%%%%%%%%%%%%%%%%%%
% Wenneker Assignment
% Structure Specification File
% Version 2.0 (12/1/2019)
%
% This template originates from:
% http://www.LaTeXTemplates.com
%
% Authors:
% Vel (vel@LaTeXTemplates.com)
% Frits Wenneker
%
% License:
% CC BY-NC-SA 3.0 (http://creativecommons.org/licenses/by-nc-sa/3.0/)
% 
%%%%%%%%%%%%%%%%%%%%%%%%%%%%%%%%%%%%%%%%%

%----------------------------------------------------------------------------------------
%	PACKAGES AND OTHER DOCUMENT CONFIGURATIONS
%----------------------------------------------------------------------------------------

\usepackage{amsmath, amsfonts, amsthm} % Math packages

\usepackage{listings} % Code listings, with syntax highlighting

\usepackage[english]{babel} % English language hyphenation

\usepackage{graphicx} % Required for inserting images
\graphicspath{{Figures/}{./}} % Specifies where to look for included images (trailing slash required)

\usepackage{booktabs} % Required for better horizontal rules in tables

\numberwithin{equation}{section} % Number equations within sections (i.e. 1.1, 1.2, 2.1, 2.2 instead of 1, 2, 3, 4)
\numberwithin{figure}{section} % Number figures within sections (i.e. 1.1, 1.2, 2.1, 2.2 instead of 1, 2, 3, 4)
\numberwithin{table}{section} % Number tables within sections (i.e. 1.1, 1.2, 2.1, 2.2 instead of 1, 2, 3, 4)

\setlength\parindent{0pt} % Removes all indentation from paragraphs

\usepackage{enumitem} % Required for list customisation
\setlist{noitemsep} % No spacing between list items

\usepackage{array}
\newcolumntype{P}[1]{>{\centering\arraybackslash}p{#1}} %Allows centering of tables

\usepackage[
backend=biber,
style=ieee,
sorting=ynt
]{biblatex}

\addbibresource{refs.bib} %Imports bibliography file

%----------------------------------------------------------------------------------------
%	DOCUMENT MARGINS
%----------------------------------------------------------------------------------------

\usepackage{geometry} % Required for adjusting page dimensions and margins

\geometry{
	paper=a4paper, % Paper size, change to letterpaper for US letter size
	top=2.5cm, % Top margin
	bottom=3cm, % Bottom margin
	left=3cm, % Left margin
	right=3cm, % Right margin
	headheight=0.75cm, % Header height
	footskip=1.5cm, % Space from the bottom margin to the baseline of the footer
	headsep=0.75cm, % Space from the top margin to the baseline of the header
	%showframe, % Uncomment to show how the type block is set on the page
}

%----------------------------------------------------------------------------------------
%	FONTS
%----------------------------------------------------------------------------------------

\usepackage[utf8]{inputenc} % Required for inputting international characters
\usepackage[T1]{fontenc} % Use 8-bit encoding

\usepackage{fourier} % Use the Adobe Utopia font for the document

%----------------------------------------------------------------------------------------
%	HEADERS AND FOOTERS
%----------------------------------------------------------------------------------------

\usepackage{scrlayer-scrpage} % Required for customising headers and footers

\ohead*{} % Right header
\ihead*{} % Left header
\chead*{} % Centre header

\ofoot*{} % Right footer
\ifoot*{} % Left footer
\cfoot*{\pagemark} % Centre footer

%----------------------------------------------------------------------------------------
%	SECTION TITLES
%----------------------------------------------------------------------------------------
 % Include the file specifying the document structure and custom commands

%----------------------------------------------------------------------------------------
%	TITLE SECTION
%----------------------------------------------------------------------------------------

\title{	
	\normalfont\normalsize
	\vspace{20pt} % Whitespace
	{\huge Julia Cheat Sheet}\\ % The assignment title
	\vspace{12pt} % Whitespace
	\rule{\linewidth}{2pt}\\ % Thick bottom horizontal rule
}

\author{\small Mo D Jabeen} % Your name

\date{\normalsize\today} % Today's date (\today) or a custom date

\begin{document}

\maketitle % Print the title

\section{General}

\begin{lstlisting}[
	caption=Function definition., % Caption above the listing
	language=python, % Use Julia functions/syntax highlighting
	frame=single, % Frame around the code listing
	showstringspaces=false, % Don't put marks in string spaces
	numbers=left, % Line numbers on left
	numberstyle=\large, % Line numbers styling
	]
	function name()
	code
	end 

	function name()
	r=3

	r,r+2 #Omit the return keyword for tuple return
	end 
\end{lstlisting}

\begin{itemize}
	\item printf for formatted prints uses the module Printf and is macro with synatax @printf	
	\item \%3f : used to show 3 sig fig
	\item \"e : scientific notation
\end{itemize}

\begin{itemize}
	\item index starts at 1 :O
	\item Strings can be indexed like arrays
	\item Combine strings using *
	\item try, catch : for error handling
\end{itemize}

\begin{lstlisting}[
	caption=Dict definition., % Caption above the listing
	language=python, % Use Julia functions/syntax highlighting
	frame=single, % Frame around the code listing
	showstringspaces=false, % Don't put marks in string spaces
	numbers=left, % Line numbers on left
	numberstyle=\large, % Line numbers styling
	]
	d = Dict(1=>"one", 2 => "two")
	d[3] = "three" # Add to the dict

	#Loops and funcs can also be placed in dicts

\end{lstlisting}

\begin{lstlisting}[
	caption=Loop/arrays definition., % Caption above the listing
	language=python, % Use Julia functions/syntax highlighting
	frame=single, % Frame around the code listing
	showstringspaces=false, % Don't put marks in string spaces
	numbers=left, % Line numbers on left
	numberstyle=\large, % Line numbers styling
	]	
	for i in 1:5 # This calls the iterate func
	println(i)
	end

	a = collect(1:20) # convert into an array

	a = map((x) -> x^2, [1,3,5,3]) # map performs func on each array element

	foreach(func, collection) #operate func on each val of collection

\end{lstlisting}

\begin{lstlisting}[
	caption= Struct definition., % Caption above the listing
	language=python, % Use Julia functions/syntax highlighting
	frame=single, % Frame around the code listing
	showstringspaces=false, % Don't put marks in string spaces
	numbers=left, % Line numbers on left
	numberstyle=\large, % Line numbers styling
	]	
	
	mutable struct name 
		string::AbstractString
		boolean::Bool 
		age::Int
		a::Array{Int,5}
	end

	newstruct = name(...)

	# Internal constructors are used to place constraints on the code

	mutable struct name
		meh::AbstractString
		numb::Int

		name(blah::AbstractString)= new(meh,4) 
		# this enforces if a struct without 
		#a number is given 4 is placed
	end

\end{lstlisting}

\begin{lstlisting}[
	caption= Tenancy operations, % Caption above the listing
	language=python, % Use Julia functions/syntax highlighting
	frame=single, % Frame around the code listing
	showstringspaces=false, % Don't put marks in string spaces
	numbers=left, % Line numbers on left
	numberstyle=\large, % Line numbers styling
	]

	x > 0 ? 1 : -1 
	# If the condition is true 1 is returned else -1 is

\end{lstlisting}

\begin{itemize}
	\item Avoid globals
	\item Locals scope is defined by code blocks ie func, loop not if
	\item Built in funcs such as iterate can be extended via multi-dispatch
	\item Use the Profiling package for measuring performance.
\end{itemize}


\section{Objects/Methods}

Structs mainly used to create new data type objects.

Inner and outer constructor methods for structs define how a new object is created based on data input.

Inner constructors enforce the same checks for multiple data types.

\begin{lstlisting}[
	caption= Constructors, % Caption above the listing
	language=python, % Use Julia functions/syntax highlighting
	frame=single, % Frame around the code listing
	showstringspaces=false, % Don't put marks in string spaces
	numbers=left, % Line numbers on left
	numberstyle=\large, % Line numbers styling
	]

	struct name{T<:Integer} <: Real 
	# <: shows all values are included in that set
	# {for arg} outer for object

		num::T
		den::T #ensures both are of type T

		#Function checks if the input numbers are empty for every object
		function name{T}(num::T, den::T) where T <: Integer

			if num == 0 && den == 0
				error("invalid")
			end
			new(num,den)
		end		
	end 
	

	name(n::Int, d::Float) = name(promote(n,d)...)
	#Outer constructor
	#Promote converts values of a single type to the same type 
	#choosing the type to work with both


	# MULTI-DISPATCH FUNCTION

	function blah(n::Int, d::Int) = println('meh')

	function blah(x::Int, y::name) = println(x*y.num)
	#This func now has two methods (multi dispatch)

\end{lstlisting}

\section{Modules}

Modules allow for better namespace control and cleaner structure.\\

They are not attached to a file, can have multiple modules in a file and multi files for the same module.\\


\textbf{using modulename:} Includes all code and exported variables.\\
\textbf{import modulename:} Includes only the code.

Can use submodules which are accessed via . operator.

\section{Differences from Python}

\begin{itemize}
	\item Use immutable Vector (same data type) instead of arrays (python would use list)
	\item Indents start with 1
	\item Include end when slicing ie [1:end] not [1:]
	\item Use [start;stop;step] format
	\item Matrix indexing creates submatrix not tuple ie X[[1:2][2:3]]
	\item To create a tuple from a matrix use (like python) X[CartesianIndex(1,1), CartesianIndex(2,3)]
	\item Variable assignment is not pointer assignment ie a= b creates new variable so they remain separate.
	\item push! is the same as append
	\item \% is remainder not modulus
	\item Int is not an unknown size its int32
	\item nothing instead of null
\end{itemize}

\section{Metaprogramming}

Julia code is represented after compiling as a data struct of type Expr.\\

\textbf{\$:} Used as interpolation for literal expression in a macro.\\
\textbf{eval:} Executes the code from Expr data type.\\
\textbf{:} Turns code into an expression (can also used quote for blocks)

Can use Expr data types as inputs to functions.

\subsection{Macros}

Compiled code as an expression not executed on runtime but during parsing.

\begin{lstlisting}[
	caption= Macro definition, % Caption above the listing
	language=python, % Use Julia functions/syntax highlighting
	frame=single, % Frame around the code listing
	showstringspaces=false, % Don't put marks in string spaces
	numbers=left, % Line numbers on left
	numberstyle=\large, % Line numbers styling
	]

	macro name()

	end

	@name() # Run using the @ operator.

\end{lstlisting}

Macros are used in code when an expression is required in multiple places before it is evaluated.

\begin{lstlisting}[
	caption= Create code, % Caption above the listing
	language=python, % Use Julia functions/syntax highlighting
	frame=single, % Frame around the code listing
	showstringspaces=false, % Don't put marks in string spaces
	numbers=left, % Line numbers on left
	numberstyle=\large, % Line numbers styling
	]

	struct MyNumber
    	x::Float64
	end
	# output

	for op = (:sin, :cos, :tan, :log, :exp)
    	@eval Base.$op(a::MyNumber) = MyNumber($op(a.x))
	end

\end{lstlisting}

\section{Concurrency}

Julia combines multi threads and cores using the same memory space as threading. CPUs using
separate memory spaces are defined as multi-processor or distributed computing.\\

\textbf{mutex:} Single lock mechanism for controlling accessing to data.\\
\textbf{semaphore:} Value signifying what are the resources being used on, for process synchronization.

Julia code tends to be purely functional and avoids mutation, generally opting for only local mutation.

If there is a shared states locks should be used or a local state (an object shared by all threads.)

A shared local state gives higher performance.

\subsection{Asynchronous}

\subsubsection{What are Tasks ?}

\textbf{Tasks} are used for asynchronous calls, ie waiting for external signals. Tasks allow switching at any
point in the execution between them and don't use extra memory space (call stack).

wait(t) - waits for the tak
\begin{lstlisting}[
	caption= Async functions, % Caption above the listing
	language=python, % Use Julia functions/syntax highlighting
	frame=single, % Frame around the code listing
	showstringspaces=false, % Don't put marks in string spaces
	numbers=left, % Line numbers on left
	numberstyle=\large, % Line numbers styling
	]

	t = @task func()
	OR
	t= @task begin ... end
	OR	
	t = Task(func)

	schedule(t,[val],error) # Allocate task to scheduler, pass val

	# if error true, val passed as an exception

	@async func() same as schedule(@task func())

	asyncmap(func,collection,ntask,batch) 
	
	# Return collection with the func executed on by ntasks.
	# Batch executes on collection in groups set by number of batch.

	yield() #Switch to scheduler to allow another task to run
	yield(t) #Switch to task t.

	Condition() #Edge triggered event source
	thread.condition - thread safe version

	Event() #Level triggered event source

	notify(condition,val,all,error) #Wake up tasks waiting for condition

	semaphore(sem_size) #counting with max at semsize

	acquire(s) #get a semaphore, blocks if none available
	release(s)

	#Use below format for locking

	lock()
	try
	...
	finally
	unlock()
	end

	bind(channel,task) 

\end{lstlisting}

\begin{lstlisting}[
	caption= Async wait functions, % Caption above the listing
	language=python, % Use Julia functions/syntax highlighting
	frame=single, % Frame around the code listing
	showstringspaces=false, % Don't put marks in string spaces
	numbers=left, % Line numbers on left
	numberstyle=\large, % Line numbers styling
	]

	wait([x])

	x {
		Channel: Wait for val
		Condition: Wait for notify 
		Process: wait for process to exit
		Task: wait for task to finish
		RawFD: change the file descriptor
	}

	#if there is no x will wait for schedule to be called

\end{lstlisting}

\subsubsection{What are Channels?}

Channels are a first in first out queue, used to connect tasks in a memory/race safe way. They can be
bounded to a task, by being placed as a parameter and therefore do not need closing.

\begin{lstlisting}[
	caption= Channel Functions, % Caption above the listing
	language=python, % Use Julia functions/syntax highlighting
	frame=single, % Frame around the code listing
	showstringspaces=false, % Don't put marks in string spaces
	numbers=left, % Line numbers on left
	numberstyle=\large, % Line numbers styling
	]

	c = Channel{Type}(limit) #limit is max number of objects in queue

	put!(channel,data) # Place data into channel
	take!(channel) #Read data from channel

	Channel(func()) - Bind a channel with a task

\end{lstlisting}

\begin{itemize}
	\item Readers will block on a take if the channel is empty
	\item Writers will block on a put if the channel is full
	\item Wait will wait until the channel has data
	\item isready test if the channel has data
\end{itemize}

\subsection{Multi threads}

Use atomic vars to ensure expected correct operation when using threads (ie for arithmetics). Careful of
finalization (tasks to clean up before garbage collection.)

\begin{lstlisting}[
	caption= threading Functions, % Caption above the listing
	language=python, % Use Julia functions/syntax highlighting
	frame=single, % Frame around the code listing
	showstringspaces=false, % Don't put marks in string spaces
	numbers=left, % Line numbers on left
	numberstyle=\large, % Line numbers styling
	]

	Threads.@threads [schedule] for ... end

	[schedule] {
		default: :dyanmic assumes equal load per thread, cant 
		guarantee thread id on an iteration
		:static one task for thread, can guarantee same id for an 
		iteration
	}

	threads.foreach(f,c,ntasks) #operate function on channel with 
	n threads

	Threads.@spawn func() #Create task and schedule to run on any 
	available thread

\end{lstlisting}

\section{Logging}

Inserting a logging statement creates an event, logging is allows for
better control and visibility than print statements.

\begin{lstlisting}[
	caption= Logging Functions, % Caption above the listing
	language=python, % Use Julia functions/syntax highlighting
	frame=single, % Frame around the code listing
	showstringspaces=false, % Don't put marks in string spaces
	numbers=left, % Line numbers on left
	numberstyle=\large, % Line numbers styling
	]

	@debug #Auto set to not output to stderr
	@info # mid level
	@warn #Higher level
	@error #highest level, generally not needed 
	# (use exceptions instead)

	@__ msg var x=var y=func() #msg in markdown

	@__ "blah $var "

\end{lstlisting}

\subsection{How do you process a log ?}

Log creation and processing are separate to allow for module level
and app level work.

\begin{lstlisting}[
	caption= Logger, % Caption above the listing
	language=python, % Use Julia functions/syntax highlighting
	frame=single, % Frame around the code listing
	showstringspaces=false, % Don't put marks in string spaces
	numbers=left, % Line numbers on left
	numberstyle=\large, % Line numbers styling
	]

	ConsoleLogger([steam,] min_level=Info) #stream can be a file io
	SimpleLogger([stream,] min_level=Info)

	global_logger(logger)
	with_logger(logger) do ... end #Local logger

\end{lstlisting}


\begin{lstlisting}[
	caption= Log filter, % Caption above the listing
	language=python, % Use Julia functions/syntax highlighting
	frame=single, % Frame around the code listing
	showstringspaces=false, % Don't put marks in string spaces
	numbers=left, % Line numbers on left
	numberstyle=\large, % Line numbers styling
	]

	disable_logging(level) #disable below this level
	should_log(logger,level) #return true if accepts this level

	handle_message(logger,level,message,val ...)

\end{lstlisting}


\begin{lstlisting}[
	caption= Example log, % Caption above the listing
	language=python, % Use Julia functions/syntax highlighting
	frame=single, % Frame around the code listing
	showstringspaces=false, % Don't put marks in string spaces
	numbers=left, % Line numbers on left
	numberstyle=\large, % Line numbers styling
	]

	# Open a textfile for writing
	io = open("log.txt", "w+")
	IOStream(<file log.txt>)
	
	# Create a simple logger
	logger = SimpleLogger(io)
	SimpleLogger(IOStream(<file log.txt>), Info, Dict{Any,Int64}())
	
	# Log a task-specific message
	with_logger(logger) do
			   @info("a context specific log message")
		   end
	
	# Write all buffered messages to the file
	flush(io)
	
	# Set the global logger to logger
	global_logger(logger)
	SimpleLogger(IOStream(<file log.txt>), Info, Dict{Any,Int64}())
	
	# This message will now also be written to the file
	@info("a global log message")
	
	# Close the file
	close(io)

	# Create a ConsoleLogger that prints any log 
	#messages with level >= Debug to stderr
	debuglogger = ConsoleLogger(stderr, Logging.Debug)

	# Enable debuglogger for a task
	with_logger(debuglogger) do
			@debug "a context specific log message"
		end

	# Set the global logger
	global_logger(debuglogger)


\end{lstlisting}

\section{Packages}

Designed around environments which can be local to inviduals or projects. Allows 
exact set of packages and their version in an environment to be controlled and 
repeated. All tracked in the manifest file.

A better version of python virtualenv.

\begin{lstlisting}[
	caption= Pkg commands, % Caption above the listing
	language=python, % Use Julia functions/syntax highlighting
	frame=single, % Frame around the code listing
	showstringspaces=false, % Don't put marks in string spaces
	numbers=left, % Line numbers on left
	numberstyle=\large, % Line numbers styling
	]

	activate [project name]

	develop --local [package name] #Allows you to use a local version of a 
	# package to develop

	free [package name] #Go back to the registered version

	get "url" or "local path"

	pin [package name] #Never update

	# Activate another persons project
	activate .
	instantiate

\end{lstlisting}

\section{Plotting}

The Julia plot pkg used is \textbf{Gadfly}, git: https://github.com/GiovineItalia/Gadfly.jl.\\

Used for plotting and visualization, using the browser to supply interactive plots, also supports
svg,png,postscript and pdf (done via draw command).

\subsection{General}

\begin{lstlisting}[
	caption= Plot commands, % Caption above the listing
	language=python, % Use Julia functions/syntax highlighting
	frame=single, % Frame around the code listing
	showstringspaces=false, % Don't put marks in string spaces
	numbers=left, % Line numbers on left
	numberstyle=\large, % Line numbers styling
	]

	plot(data, elements ....; mapping)
	#data is a dataframe
	#elements are the plot ie x= :col, color = blue etc
	#mapping is the type of graph ie Geom.point, Geom.line

	map(display,plot) 
	#Or can use REPL to output to default media viewer (browser)

	plot(func(),lower,upper,elements;mapping)
	#lower upper are x bounds

	plot(Vector{funcs()},lower,upper,elements;mapping)
	plot(func(),xmin,xmax,ymin,ymax,elements;mapping)

	# Can also add elements to plot via push!

\end{lstlisting}

\textit{Widerform is when cols use the sames measurment type but are grouped in cols.}\\

Elements have a Scale (continous or discrete) and Guide class:\\

Types of Scale:
\begin{itemize}
	\item Scale.x\_continous(format=, minvalue,maxvalue)
	\item Scale.x\_discrete
	\item Scale.xlog10 : Added as elements to change entire plot
	\item Scale.ylog10
\end{itemize}

\begin{lstlisting}[
	caption= Scale examples, % Caption above the listing
	language=python, % Use Julia functions/syntax highlighting
	frame=single, % Frame around the code listing
	showstringspaces=false, % Don't put marks in string spaces
	numbers=left, % Line numbers on left
	numberstyle=\large, % Line numbers styling
	]

	p2 = plot(mammals, x=:Body, y=:Brain, label=:Mammal, 
		Geom.point, Geom.label,
		Scale.x_log10, Scale.y_log10)

	p3= plot(Diamonds, x=:Price, y=:Carat, 
		Geom.histogram2d(xbincount=25, ybincount=25),
    	Scale.x_continuous(format=:engineering))

\end{lstlisting}

Types of Guides:
\begin{itemize}
	\item Annotation
	\item Keys : colour, shape or size
	\item xrug,yrug : show distribution of points on axis
	\item ticks : show plot lines
\end{itemize}

\begin{lstlisting}[
	caption= Guide examples, % Caption above the listing
	language=python, % Use Julia functions/syntax highlighting
	frame=single, % Frame around the code listing
	showstringspaces=false, % Don't put marks in string spaces
	numbers=left, % Line numbers on left
	numberstyle=\large, % Line numbers styling
	]

	pb = plot(iris, x=:SepalLength, y=:PetalLength, color=:Species, 
		Geom.point,
        Guide.colorkey(title="Iris", pos=[0.05w,-0.28h]) )

	plot(dataset("ggplot2", "diamonds"), x="Price", Geom.histogram,
		Guide.title("Diamond Price Distribution"))

\end{lstlisting}

\subsection{Compositing}

Types:
\begin{itemize}
	\item Facets: Share same axis dims but seperate
	\item Stack: Diff datasets and axis but together
	\item Layers: Single plot
\end{itemize}

\subsubsection{Layers}

\begin{lstlisting}[
	caption= Layer examples, % Caption above the listing
	language=python, % Use Julia functions/syntax highlighting
	frame=single, % Frame around the code listing
	showstringspaces=false, % Don't put marks in string spaces
	numbers=left, % Line numbers on left
	numberstyle=\large, % Line numbers styling
	]

	l = layer(data,elements;mapping) 
	# Can not include Scale,coordinates or guides
	#use plot for that

	plot(layer1,layer2..)

	plot(data
		layer(elements;mapping)
		layer(...))

\end{lstlisting}

%----------------------------------------------------------------------------------------
%	FIGURE EXAMPLE
%----------------------------------------------------------------------------------------

% \begin{figure}[h] % [h] forces the figure to be output where it is defined in the code (it suppresses floating)
% 	\centering
% 	\includegraphics[width=0.5\columnwidth]{IMAGE_NAME.jpg} % Example image
% 	\caption{European swallow.}
% \end{figure}

%----------------------------------------------------------------------------------------
% MATH EXAMPLES
%----------------------------------------------------------------------------------------

% \begin{align} 
% 	\label{eq:bayes}
% 	\begin{split}
% 		P(A|B) = \frac{P(B|A)P(A)}{P(B)}
% 	\end{split}					
% \end{align}

%----------------------------------------------------------------------------------------
%	LIST EXAMPLES
%----------------------------------------------------------------------------------------

% \begin{itemize}
% 	\item First item in a list 
% 		\begin{itemize}
% 		\item First item in a list 
% 			\begin{itemize}
% 			\item First item in a list 
% 			\item Second item in a list 
% 			\end{itemize}
% 		\item Second item in a list 
% 		\end{itemize}
% 	\item Second item in a list 
% \end{itemize}

%------------------------------------------------

% \subsection{Numbered List}

% \begin{enumerate}
% 	\item First item in a list 
% 	\item Second item in a list 
% 	\item Third item in a list
% \end{enumerate}

%----------------------------------------------------------------------------------------
%	TABLE EXAMPLE
%----------------------------------------------------------------------------------------

% \section{Interpreting a Table}

% \begin{table}[h] % [h] forces the table to be output where it is defined in the code (it suppresses floating)
% 	\centering % Centre the table
% 	\begin{tabular}{l l l}
% 		\toprule
% 		\textit{Per 50g} & \textbf{Pork} & \textbf{Soy} \\
% 		\midrule
% 		Energy & 760kJ & 538kJ\\
% 		Protein & 7.0g & 9.3g\\
% 		\bottomrule
% 	\end{tabular}
% 	\caption{Sausage nutrition.}
% \end{table}

%----------------------------------------------------------------------------------------
%	CODE LISTING EXAMPLE
%----------------------------------------------------------------------------------------

% \lstinputlisting[
% 	caption=Luftballons Perl Script., % Caption above the listing
% 	label=lst:luftballons, % Label for referencing this listing
% 	language=Perl, % Use Perl functions/syntax highlighting
% 	frame=single, % Frame around the code listing
% 	showstringspaces=false, % Don't put marks in string spaces
% 	numbers=left, % Line numbers on left
% 	numberstyle=\tiny, % Line numbers styling
% 	]{luftballons.pl}

%------------------------------------------------

% \begin{lstlisting}[
% 	caption= [NAME], % Caption above the listing
% 	language=python, % Use Julia functions/syntax highlighting
% 	frame=single, % Frame around the code listing
% 	showstringspaces=false, % Don't put marks in string spaces
% 	numbers=left, % Line numbers on left
% 	numberstyle=\large, % Line numbers styling
% 	]

% \end{lstlisting}

\end{document}