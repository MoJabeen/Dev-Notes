% Mo Jabeen Template for docs 

\documentclass[11pt]{scrartcl} % Font size

\input{structure.tex} % Include the file specifying the document structure and custom commands

%----------------------------------------------------------------------------------------
%	TITLE SECTION
%----------------------------------------------------------------------------------------

\title{	
	\normalfont\normalsize
	\vspace{20pt} % Whitespace
	{\huge Database Fundamentals}\\ % The assignment title
	\vspace{12pt} % Whitespace
	\rule{\linewidth}{2pt}\\ % Thick bottom horizontal rule
}

\author{\small Mo D Jabeen} % Your name

\date{\normalsize\today} % Today's date (\today) or a custom date

\begin{document}

\maketitle % Print the title

\section{Evolution of database design}

\subsection{What has triggered the evolution in database?}

\begin{itemize}
	\item Business need to be agile, hypothesis/business models need to be tested fast and then
	decisions made if a pivot is required. Market insights should allow for quick changes to products/operations.
	\item CPU improvement is decelerating, and parallelism is increasing
\end{itemize}

\subsection{What is data intensive application?}

A data intensive application primary challenge is the use of data (storage, transformation, transmission etc)
this is primary bottle neck whereas in compute intensive apps CPU cycles is the bottle neck.

\section{Fundamental metrics}

\textbf{Reliability:} Tolerate hardware,software and human faults.\\
\textbf{Scalability:} Maintain load and performance as quantities increase.\\
\textbf{Maintainability:} Operability, simplicity and evolvability. (Ease of understanding).

\section{Database Systems}

\subsection{What are some elements of a database system?}

\begin{itemize}
	\item Cache - Results of expensive operations
	\item Stream processing - Asynchronous processes messaging
	\item Batch processing - Crunch a large amount of accumulated data
	\item Message queue - Hold data for use with other processes
\end{itemize}

\subsection{Which tool should you use?}

No single tools fits all applications, instead the work should be broken into tasks
and then the most appropriate/effective tool used. 

\textit{Example: Caching - Memcached}

\section{Internal vs External}

\subsection{In code data structures}

Data structures in code are should be structured and used differently to external databases. In code
data ie for OOP should be based around their use in logic. Databases can be used by multiple separate
processes whereas data structs should only be used by local code.\\

Data structs should:

\begin{itemize}
	\item Be limited in size
	\item Not generally used for concurrent programs
	\item Not tied to ACID (Atomic,consistent,isolated and durable)
	\item Fast
\end{itemize}

\section{Understanding Text}

\subsubsection{subsubsection}

%----------------------------------------------------------------------------------------
%	FIGURE EXAMPLE
%----------------------------------------------------------------------------------------

% \begin{figure}[h] % [h] forces the figure to be output where it is defined in the code (it suppresses floating)
% 	\centering
% 	\includegraphics[width=0.5\columnwidth]{IMAGE_NAME.jpg} % Example image
% 	\caption{European swallow.}
% \end{figure}

%----------------------------------------------------------------------------------------
% MATH EXAMPLES
%----------------------------------------------------------------------------------------

% \begin{align} 
% 	\label{eq:bayes}
% 	\begin{split}
% 		P(A|B) = \frac{P(B|A)P(A)}{P(B)}
% 	\end{split}					
% \end{align}

%----------------------------------------------------------------------------------------
%	LIST EXAMPLES
%----------------------------------------------------------------------------------------

% \begin{itemize}
% 	\item First item in a list 
% 		\begin{itemize}
% 		\item First item in a list 
% 			\begin{itemize}
% 			\item First item in a list 
% 			\item Second item in a list 
% 			\end{itemize}
% 		\item Second item in a list 
% 		\end{itemize}
% 	\item Second item in a list 
% \end{itemize}

%------------------------------------------------

% \subsection{Numbered List}

% \begin{enumerate}
% 	\item First item in a list 
% 	\item Second item in a list 
% 	\item Third item in a list
% \end{enumerate}

%----------------------------------------------------------------------------------------
%	TABLE EXAMPLE
%----------------------------------------------------------------------------------------

% \section{Interpreting a Table}

% \begin{table}[h] % [h] forces the table to be output where it is defined in the code (it suppresses floating)
% 	\centering % Centre the table
% 	\begin{tabular}{l l l}
% 		\toprule
% 		\textit{Per 50g} & \textbf{Pork} & \textbf{Soy} \\
% 		\midrule
% 		Energy & 760kJ & 538kJ\\
% 		Protein & 7.0g & 9.3g\\
% 		\bottomrule
% 	\end{tabular}
% 	\caption{Sausage nutrition.}
% \end{table}

%----------------------------------------------------------------------------------------
%	CODE LISTING EXAMPLE
%----------------------------------------------------------------------------------------

% \begin{lstlisting}[
% 	caption= Macro definition, % Caption above the listing
% 	language=python, % Use Julia functions/syntax highlighting
% 	frame=single, % Frame around the code listing
% 	showstringspaces=false, % Don't put marks in string spaces
% 	numbers=left, % Line numbers on left
% 	numberstyle=\large, % Line numbers styling
% 	]

% 	CODE

% \end{lstlisting}

%----------------------------------------------------------------------------------------
%	CODE LISTING FILE EXAMPLE
%----------------------------------------------------------------------------------------

% \lstinputlisting[
% 	caption=Luftballons Perl Script., % Caption above the listing
% 	label=lst:luftballons, % Label for referencing this listing
% 	language=Perl, % Use Perl functions/syntax highlighting
% 	frame=single, % Frame around the code listing
% 	showstringspaces=false, % Don't put marks in string spaces
% 	numbers=left, % Line numbers on left
% 	numberstyle=\tiny, % Line numbers styling
% 	]{luftballons.pl}

%------------------------------------------------

\end{document}