% Mo Jabeen Template for docs 

\documentclass[11pt]{scrartcl} % Font size

%%%%%%%%%%%%%%%%%%%%%%%%%%%%%%%%%%%%%%%%%
% Wenneker Assignment
% Structure Specification File
% Version 2.0 (12/1/2019)
%
% This template originates from:
% http://www.LaTeXTemplates.com
%
% Authors:
% Vel (vel@LaTeXTemplates.com)
% Frits Wenneker
%
% License:
% CC BY-NC-SA 3.0 (http://creativecommons.org/licenses/by-nc-sa/3.0/)
% 
%%%%%%%%%%%%%%%%%%%%%%%%%%%%%%%%%%%%%%%%%

%----------------------------------------------------------------------------------------
%	PACKAGES AND OTHER DOCUMENT CONFIGURATIONS
%----------------------------------------------------------------------------------------

\usepackage{amsmath, amsfonts, amsthm} % Math packages

\usepackage{listings} % Code listings, with syntax highlighting

\usepackage[english]{babel} % English language hyphenation

\usepackage{graphicx} % Required for inserting images
\graphicspath{{Figures/}{./}} % Specifies where to look for included images (trailing slash required)

\usepackage{booktabs} % Required for better horizontal rules in tables

\numberwithin{equation}{section} % Number equations within sections (i.e. 1.1, 1.2, 2.1, 2.2 instead of 1, 2, 3, 4)
\numberwithin{figure}{section} % Number figures within sections (i.e. 1.1, 1.2, 2.1, 2.2 instead of 1, 2, 3, 4)
\numberwithin{table}{section} % Number tables within sections (i.e. 1.1, 1.2, 2.1, 2.2 instead of 1, 2, 3, 4)

\setlength\parindent{0pt} % Removes all indentation from paragraphs

\usepackage{enumitem} % Required for list customisation
\setlist{noitemsep} % No spacing between list items

\usepackage{array}
\newcolumntype{P}[1]{>{\centering\arraybackslash}p{#1}} %Allows centering of tables

\usepackage[
backend=biber,
style=ieee,
sorting=ynt
]{biblatex}

\addbibresource{refs.bib} %Imports bibliography file

%----------------------------------------------------------------------------------------
%	DOCUMENT MARGINS
%----------------------------------------------------------------------------------------

\usepackage{geometry} % Required for adjusting page dimensions and margins

\geometry{
	paper=a4paper, % Paper size, change to letterpaper for US letter size
	top=2.5cm, % Top margin
	bottom=3cm, % Bottom margin
	left=3cm, % Left margin
	right=3cm, % Right margin
	headheight=0.75cm, % Header height
	footskip=1.5cm, % Space from the bottom margin to the baseline of the footer
	headsep=0.75cm, % Space from the top margin to the baseline of the header
	%showframe, % Uncomment to show how the type block is set on the page
}

%----------------------------------------------------------------------------------------
%	FONTS
%----------------------------------------------------------------------------------------

\usepackage[utf8]{inputenc} % Required for inputting international characters
\usepackage[T1]{fontenc} % Use 8-bit encoding

\usepackage{fourier} % Use the Adobe Utopia font for the document

%----------------------------------------------------------------------------------------
%	HEADERS AND FOOTERS
%----------------------------------------------------------------------------------------

\usepackage{scrlayer-scrpage} % Required for customising headers and footers

\ohead*{} % Right header
\ihead*{} % Left header
\chead*{} % Centre header

\ofoot*{} % Right footer
\ifoot*{} % Left footer
\cfoot*{\pagemark} % Centre footer

%----------------------------------------------------------------------------------------
%	SECTION TITLES
%----------------------------------------------------------------------------------------
 % Include the file specifying the document structure and custom commands

%----------------------------------------------------------------------------------------
%	TITLE SECTION
%----------------------------------------------------------------------------------------


\title{	
	\normalfont\normalsize
	\vspace{20pt} % Whitespace
	{\huge Golang Cheat Sheet}\\ % The meh
	\vspace{12pt} % Whitespace
	\rule{\linewidth}{2pt}\\ % Thick bottom horizontal rule
}

\author{\small Dainish Jabeen} % Your name

\date{\normalsize\today} % Today's date (\today) or a custom date

\begin{document}

\maketitle % Print the title

\section{Packages}

Go uses packages, which can contain multiple files. \textbf{The app will start running in the main 
application.}\\

Names exported outside the packages, must use a Capital letter.

\begin{lstlisting}[
	caption= Golang basics, % Caption above the listing
	language=Go, % Use Julia functions/syntax highlighting
	frame=single, % Frame around the code listing
	showstringspaces=false, % Don't put marks in string spaces
	numbers=left, % Line numbers on left
	numberstyle=\large, % Line numbers styling
	]

	package main

	import "fmt"

	func main(){
		fmt.Println("Hello World")
	}

\end{lstlisting}

\newpage

\section{General}

\subsection{Types}

\begin{lstlisting}[
	caption= Golang types, % Caption above the listing
	language=Go, % Use Julia functions/syntax highlighting
	frame=single, % Frame around the code listing
	showstringspaces=false, % Don't put marks in string spaces
	numbers=left, % Line numbers on left
	numberstyle=\large, % Line numbers styling
	]

	:= //Declare and initialize non explicit type
	>> //Shift bitwise right
	<<  //Shift bitwise left

	//ARRAY
	name []string 
	var := []string("blah","meh")

	//MAPS
	map[key type]val type //Dict has to be made
	m := make(map[string]String)

	elem, ok = m[key] // check key exists

	// Initialization

	var i int // initializes as 0

	// Constants

	const(
		x=1 // the type of this can change on context
	)

	p := &i //point to i
	*p //value of i, changes will change also change i

	type Name struct{
		x int
		y int
	}

	// Pointers to structs

	v := StructName{1,2}

	p = &v
	p.x = //Will change the value of v

	// Struct constructors

	v := StructName{x:1} //Others members made 0

	//SLICES

	//Slices acts as pointers

	a[1:] // slice to end
	s := a[:3] // slice start to 3

	cap(s) // Capacity, elements in underlying array

	// Dynamically size arrays

	a := make(type,len,cap)
	append(arr,val)
	
\end{lstlisting}

\newpage

\subsection{Functions}

\begin{lstlisting}[
	caption= Functions, % Caption above the listing
	language=Go, % Use Julia functions/syntax highlighting
	frame=single, % Frame around the code listing
	showstringspaces=false, % Don't put marks in string spaces
	numbers=left, % Line numbers on left
	numberstyle=\large, % Line numbers styling
	]

	func Name(name type) type{}

	//FUNC PARAMS

	func name(x int,y int)
	func name(x,y int)

	//NAKED RETURN

	func() (x,y int) {
		x:=1
		y:=2
		return
	} // Will return x and y 

\end{lstlisting}

\subsubsection{Funcs as params}

\begin{lstlisting}[
	caption= Params, % Caption above the listing
	language=Go, % Use Julia functions/syntax highlighting
	frame=single, % Frame around the code listing
	showstringspaces=false, % Don't put marks in string spaces
	numbers=left, % Line numbers on left
	numberstyle=\large, % Line numbers styling
	]

	func compute(fn func(float64, float64) float64) float64 {
	return fn(3, 4)
	}

	func main() {
		hypot := func(x, y float64) float64 {
			return math.Sqrt(x*x + y*y)
		}
		fmt.Println(hypot(5, 12))

		fmt.Println(compute(hypot))
		fmt.Println(compute(math.Pow))
	}

\end{lstlisting}

\subsection{Control}

\begin{lstlisting}[
	caption= Control, % Caption above the listing
	language=Go, % Use Julia functions/syntax highlighting
	frame=single, % Frame around the code listing
	showstringspaces=false, % Don't put marks in string spaces
	numbers=left, % Line numbers on left
	numberstyle=\large, % Line numbers styling
	]

	for i:=0;i<10;i++{}

	for x<100 {} //Same as while loop

	if statement; cond {}

	switch statement; val {

		case x: //x same as val == x

		case y: //y same as val == y

		default:
	}

	defer expr //execute expr at the end of func, can stack defers

	for i,v := range arr {} // Loops through array (can do _,v or i,_)

\end{lstlisting}

%----------------------------------------------------------------------------------------
%	FIGURE EXAMPLE
%----------------------------------------------------------------------------------------

% \begin{figure}[h] % [h] forces the figure to be output where it is defined in the code (it suppresses floating)
% 	\centering
% 	\includegraphics[width=0.5\columnwidth]{IMAGE_NAME.jpg} % Example image
% 	\caption{European swallow.}
% \end{figure}

%----------------------------------------------------------------------------------------
% MATH EXAMPLES
%----------------------------------------------------------------------------------------

% \begin{align} 
% 	\label{eq:bayes}
% 	\begin{split}
% 		P(A|B) = \frac{P(B|A)P(A)}{P(B)}
% 	\end{split}					
% \end{align}

%----------------------------------------------------------------------------------------
%	LIST EXAMPLES
%----------------------------------------------------------------------------------------

% \begin{itemize}
% 	\item First item in a list 
% 		\begin{itemize}
% 		\item First item in a list 
% 			\begin{itemize}
% 			\item First item in a list 
% 			\item Second item in a list 
% 			\end{itemize}
% 		\item Second item in a list 
% 		\end{itemize}
% 	\item Second item in a list 
% \end{itemize}

%------------------------------------------------

% \subsection{Numbered List}

% \begin{enumerate}
% 	\item First item in a list 
% 	\item Second item in a list 
% 	\item Third item in a list
% \end{enumerate}

%----------------------------------------------------------------------------------------
%	TABLE EXAMPLE
%----------------------------------------------------------------------------------------

% \section{Interpreting a Table}

% \begin{table}[h] % [h] forces the table to be output where it is defined in the code (it suppresses floating)
% 	\centering % Centre the table
% 	\begin{tabular}{l l l}
% 		\toprule
% 		\textit{Per 50g} & \textbf{Pork} & \textbf{Soy} \\
% 		\midrule
% 		Energy & 760kJ & 538kJ\\
% 		Protein & 7.0g & 9.3g\\
% 		\bottomrule
% 	\end{tabular}
% 	\caption{Sausage nutrition.}
% \end{table}

%----------------------------------------------------------------------------------------
%	CODE LISTING EXAMPLE
%----------------------------------------------------------------------------------------

% \begin{lstlisting}[
% 	caption= Macro definition, % Caption above the listing
% 	language=python, % Use Julia functions/syntax highlighting
% 	frame=single, % Frame around the code listing
% 	showstringspaces=false, % Don't put marks in string spaces
% 	numbers=left, % Line numbers on left
% 	numberstyle=\large, % Line numbers styling
% 	]

% 	CODE

% \end{lstlisting}

%----------------------------------------------------------------------------------------
%	CODE LISTING FILE EXAMPLE
%----------------------------------------------------------------------------------------

% \lstinputlisting[
% 	caption=Luftballons Perl Script., % Caption above the listing
% 	label=lst:luftballons, % Label for referencing this listing
% 	language=Perl, % Use Perl functions/syntax highlighting
% 	frame=single, % Frame around the code listing
% 	showstringspaces=false, % Don't put marks in string spaces
% 	numbers=left, % Line numbers on left
% 	numberstyle=\tiny, % Line numbers styling
% 	]{luftballons.pl}

%----------------------------------------------------------------------------------------
%	BIB EXAMPLE
%----------------------------------------------------------------------------------------

% Using \texttt{biblatex} you can display a bibliography divided into sections, depending on citation type. 
% Let's cite! Einstein's journal paper \cite{einstein} and Dirac's book \cite{dirac} are physics-related items. 
% Next, \textit{The \LaTeX\ Companion} book \cite{latexcompanion}, Donald Knuth's website \cite{knuthwebsite}, \textit{The Comprehensive Tex Archive Network} (CTAN) \cite{ctan} are \LaTeX-related items; but the others, Donald Knuth's items, \cite{knuth-fa,knuth-acp} are dedicated to programming. 

% \medskip

% \printbibliography[
% heading=bibintoc,
% title={Whole bibliography}
% ] %Prints the entire bibliography with the title "Whole bibliography"

% %Filters bibliography
% \printbibliography[heading=subbibintoc,type=article,title={Articles only}]
% \printbibliography[type=book,title={Books only}]

% \printbibliography[keyword={physics},title={Physics-related only}]
% \printbibliography[keyword={latex},title={\LaTeX-related only}]

\end{document}