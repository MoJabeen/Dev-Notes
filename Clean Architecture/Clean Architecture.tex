% Mo Jabeen Template for docs 

\documentclass[11pt]{scrartcl} % Font size

\input{structure.tex} % Include the file specifying the document structure and custom commands

%----------------------------------------------------------------------------------------
%	TITLE SECTION
%----------------------------------------------------------------------------------------


\title{	
	\normalfont\normalsize
	\vspace{20pt} % Whitespace
	{\huge Clean Architecture}\\ % The meh
	\vspace{12pt} % Whitespace
	\rule{\linewidth}{2pt}\\ % Thick bottom horizontal rule
}

\author{\small Dainish Jabeen} % Your name

\date{\normalsize\today} % Today's date (\today) or a custom date

\begin{document}

\maketitle % Print the title

The goal of software architecture is to minimize the human resources
required to build and maintain the system.

There are two values stakeholders care about: the behavior and structure
of a system.

\textbf{Behavior:} Make the machine fulfill the users requirements.

\textbf{Architecture:} Soft ware should be easy to change, the
difficulty of the change should be proportional to the scope of change
and not the ``shape''. Shape being the type of change requested.

\textbf{Eisenhower matrix:} A digram showing the combination of
important and urgent. The conclusion is that generally things that are
urgent should not be important and things that are important should not
be urgent.

Three big areas of architecture: function, separation of components and
data management.

\section{Object orientated}

First area thought to be introduced is Encapsulation of data and
functions.

Data should be kept within concepts and only accessed through this
concept.

Good encapsulation is when the users of a program have or need no
knowledge of the implementation of the data structure of function.

\begin{itemize}
\item
  OO does not require perfect encapsulation, modern languages have
  declaration and definition of the classes tied together needing
  knowledge of one and other to use them.
\end{itemize}

\textbf{Inheritance:} Able to reuse classes and modules in different
scopes.

Encapsulation can exist without OO so this is not its feature.

\subsection{What is polymorphism ?}

At its core polymorphism is pointers to functions. Allowing the pointer
to dictate based on the given parameter the functions behavior.

This can be done without OO, however it makes it much safer and more
convenient.

\subsection{How did OO effect the flow of dependency (Dependency
inversion)}

Prior to OO flow of control had to match the direction of dependency.

\begin{itemize}
\item
  All higher level functions are dependant on main and the flow of
  control stems down to them from main.
\end{itemize}

The dependency can be inverted using OO and interfaces; giving the
architect control over the dependency tree. \textbf{Control order does
not have to match dependency}

\emph{Green arrow = Control , Red arrow = dependency}

\begin{itemize}
\item
  The UI can depend on the business rules and the control flow stem from
  the rules.
\end{itemize}

The main thing OO introduces is a safe effective method of controlling
source code dependencies !!

Vital to allow modularity and plugin nature of development.

\#~Functional Programming

Functional programs variable are not mutable, they do not vary.

\textbf{All concurrent update problems derive from mutable variables}

\subsection{How do you utilize immutable variables}

If there was infinite resources (memory and processing) using immutable
variables would be an option. Instead need to split the program into
mutable and immutable components.

The immutable components will feed the mutable components which then use
transactional memory.


Transactional memory acts as disk database with safety measures against
race conditions; locking mechanisms on reading and writing.

\subsection{What is event sourcing ?}

Store transactions instead of states to avoid mutable variables. The
state is then calculated by summing transactions.

\section{Components}

Well designed components are independently deployable and therefore
independently developable.

Three principles associated with component design :

Module managment tools: Maven, Leiningen and RVM.

\subsection{Reuse/Release equivalence principle}

Should be an overarching theme of the modules inside a component.

To allow effective developing and reuse of a component there should be
scheduled releases of new versions. And therefore the classes and
components should be releasable together as a unit.

\subsection{Common closure principle}

Gather classes/modules into a component that change change for the same
reasons at the same time.

Much easier to change related classes in a single component than across
many components.

\subsection{Common reuse principle}

Dont force users of a component to depend on a code they dont need.
Places reused classes and modules together into a component. There
should be a high dependency in a component between classes and modules.

If dependant on a component better to be completely dependant on the
entire component.

\subsection{How do you balance the three principles}

It is dependant on the needs of the software system and there will be
tension between all three.

i.e Early on development CCP is more important than REP ! As development
is more important than reuse.


\section{Component Cohesion}

\subsection{Acyclic dependencies principle}

No cycles in the component dependency graph.

Released components allow devs to work in isolated teams and decide
which version to integrate with their dependant component.

If there is a dependency cycle between components multiple components
will be forced to use the same version of a dependant components
disrupting the isolated teams.

To break a cycle create an interface component which will inverse the
dependency. Or create another component between.

\subsection{Stable dependency principle}

Do not make easy to change modules be depended on by hard to change
modules.

A method to make software stable and therefore hard to change is to have
a lot of software dependant on it.

\subsubsection{How do you measure stability ?}

By the number of dependencies:

Fan in: Incoming dependencies (Increasing this number is good for
stability) Fan out: Outgoing dependencies

Instability : I = Fan out/(Fan in + Fan out), I=0 is maximum stability
and I=1 is max unstable.

\subsection{Stable abstraction principle}

A component should be as abstract as it is stable.

A stable component should be abstracted so it can be \textbf{extended},
it should be hard to change however being extendable does not effect
this.

\subsubsection{How can you measure abstractness}

Abstractness = Number of abstract classes and interfaces / number of
classes in component


\subsubsection{Metric to determine good design}

Distance from main sequence :

D = \textbar{} A + I -1 \textbar{}

\section{Architecture}

Architecture is about deployment, maintenance and ongoing development.
Not directly linked with behavior or operation however it can aid in
this.

The focus is on reliability and development speed not performance.

As many options as possible should be kept open for as long as possible
by good design. Until you are in the most optimum position to make a
judgment on an option (database, language, hardware etc)

The concept is to ensure policy is completely decoupled from details.
%----------------------------------------------------------------------------------------
%	FIGURE EXAMPLE
%----------------------------------------------------------------------------------------

% \begin{figure}[h] % [h] forces the figure to be output where it is defined in the code (it suppresses floating)
% 	\centering
% 	\includegraphics[width=0.5\columnwidth]{IMAGE_NAME.jpg} % Example image
% 	\caption{European swallow.}
% \end{figure}

%----------------------------------------------------------------------------------------
% MATH EXAMPLES
%----------------------------------------------------------------------------------------

% \begin{align} 
% 	\label{eq:bayes}
% 	\begin{split}
% 		P(A|B) = \frac{P(B|A)P(A)}{P(B)}
% 	\end{split}					
% \end{align}

%----------------------------------------------------------------------------------------
%	LIST EXAMPLES
%----------------------------------------------------------------------------------------

% \begin{itemize}
% 	\item First item in a list 
% 		\begin{itemize}
% 		\item First item in a list 
% 			\begin{itemize}
% 			\item First item in a list 
% 			\item Second item in a list 
% 			\end{itemize}
% 		\item Second item in a list 
% 		\end{itemize}
% 	\item Second item in a list 
% \end{itemize}

%------------------------------------------------

% \subsection{Numbered List}

% \begin{enumerate}
% 	\item First item in a list 
% 	\item Second item in a list 
% 	\item Third item in a list
% \end{enumerate}

%----------------------------------------------------------------------------------------
%	TABLE EXAMPLE
%----------------------------------------------------------------------------------------

% \section{Interpreting a Table}

% \begin{table}[h] % [h] forces the table to be output where it is defined in the code (it suppresses floating)
% 	\centering % Centre the table
% 	\begin{tabular}{l l l}
% 		\toprule
% 		\textit{Per 50g} & \textbf{Pork} & \textbf{Soy} \\
% 		\midrule
% 		Energy & 760kJ & 538kJ\\
% 		Protein & 7.0g & 9.3g\\
% 		\bottomrule
% 	\end{tabular}
% 	\caption{Sausage nutrition.}
% \end{table}

%----------------------------------------------------------------------------------------
%	CODE LISTING EXAMPLE
%----------------------------------------------------------------------------------------

% \begin{lstlisting}[
% 	caption= Macro definition, % Caption above the listing
% 	language=python, % Use Julia functions/syntax highlighting
% 	frame=single, % Frame around the code listing
% 	showstringspaces=false, % Don't put marks in string spaces
% 	numbers=left, % Line numbers on left
% 	numberstyle=\large, % Line numbers styling
% 	]

% 	CODE

% \end{lstlisting}

%----------------------------------------------------------------------------------------
%	CODE LISTING FILE EXAMPLE
%----------------------------------------------------------------------------------------

% \lstinputlisting[
% 	caption=Luftballons Perl Script., % Caption above the listing
% 	label=lst:luftballons, % Label for referencing this listing
% 	language=Perl, % Use Perl functions/syntax highlighting
% 	frame=single, % Frame around the code listing
% 	showstringspaces=false, % Don't put marks in string spaces
% 	numbers=left, % Line numbers on left
% 	numberstyle=\tiny, % Line numbers styling
% 	]{luftballons.pl}

%----------------------------------------------------------------------------------------
%	BIB EXAMPLE
%----------------------------------------------------------------------------------------

% Using \texttt{biblatex} you can display a bibliography divided into sections, depending on citation type. 
% Let's cite! Einstein's journal paper \cite{einstein} and Dirac's book \cite{dirac} are physics-related items. 
% Next, \textit{The \LaTeX\ Companion} book \cite{latexcompanion}, Donald Knuth's website \cite{knuthwebsite}, \textit{The Comprehensive Tex Archive Network} (CTAN) \cite{ctan} are \LaTeX-related items; but the others, Donald Knuth's items, \cite{knuth-fa,knuth-acp} are dedicated to programming. 

% \medskip

% \printbibliography[
% heading=bibintoc,
% title={Whole bibliography}
% ] %Prints the entire bibliography with the title "Whole bibliography"

% %Filters bibliography
% \printbibliography[heading=subbibintoc,type=article,title={Articles only}]
% \printbibliography[type=book,title={Books only}]

% \printbibliography[keyword={physics},title={Physics-related only}]
% \printbibliography[keyword={latex},title={\LaTeX-related only}]

\end{document}